% !TEX root = ../lust-1.tex
%
\chapter{Lernen in Gruppen}
\label{sec:lernen-in-gruppen}

\paragraph{Teil 5 der Hausarbeit}
Schildern Sie Ihre eigenen Erfahrungen mit den verschiedenen Kommunikationsmustern. Wo sind die verschiedenen Muster in Ihrem Leben schon einmal aufgetaucht (dies muss nicht zwingend in Lerngruppen sein)? Welche Vorteile hatten die Kommunikationsmuster in der jeweiligen Situation? Was hat vielleicht Probleme bereitet? \\[0.4em]

Das Kommunikationsmuster “Kette” tritt beispielsweise bei Gruppenarbeiten in der Uni auf. Jedes Gruppenmitglied ist verantwortlich für einen Teil der Arbeit und besitzt selbst Schnittstellen zu angrenzenden Teilbereichen. Die Problematik hierbei ist, dass nicht zwangsläufig immer parallel gearbeitet werden kann, da unter Umständen ein Mitglied auf die zuarbeit eines anderen warten muss. Außerdem propagieren in der Mitte getroffene Entscheidungen nur langsam an die Ränder. Es entstehen also längere Laufwege.

Der Vorsitz der Fachschaftsvertretung steht  als Koordinator im Mittelpunkt. Zu vergleichen ist dieses Muster also mit dem des “Sterns”. Die Arbeiten werden zentral koordiniert und verteilt. Dies kann jedoch unter Umständen zu Unzufriedenheit der anderen Gruppenmitgliedern führen, da die Gefahr besteht, dass ein Gefühl von ungerecht verteilter Arbeit entsteht.

Der “Kreis” ist vergleichbar mit einer Software Entwicklungsabteilung. Herrscht hier eine agile Arbeitsweise, sind alle Gruppenmitglieder gleichermaßen an dem Erstellen neuer Tasks und dem Abarbeiten von Aufgaben beteiligt. Entscheidungen können durch den direkten Austausch effizient kommuniziert und umgesetzt werden.Vorteil dabei ist, dass die geleistete Arbeit direkt messbar wird.
