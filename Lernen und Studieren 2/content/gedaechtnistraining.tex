% !TEX root = ../lust-1.tex
%
\chapter{Gedächtnistraining}
\label{sec:gedaechtnistraining}

\paragraph{Teil 4 der Hausarbeit}
Überlegen Sie sich zu 3 der 5 Mnemotechniken Beispiele und erläutern Sie die Methoden anhand dieser Beispiele. Was ist Ihre „Lieblingsmethode“ und warum? Welche Methode gefällt Ihnen nicht so sehr und warum? \\[0.4em]

Als Beispiel für die Assoziationsmethode dient das Einprägen von Namen neuer Bekanntschaften. Hier kann der Name beispielsweise mit einer persönlichen Eigenschaft verknüpft werden: Der Brillen-tragende Benno.
Alternativ kann auch der Ort als Assoziation dienen: Benno aus dem Büro nebenan. Der große Vorteil dieser Methode ist, dass zusätzlich zur eigentlichen Information noch meta-Informationen (wie bspw. der Ort) gelernt werden.

Für das Fremdwort “Ambivalenz” bietet sich die Ersatzwortmethode an. Als Merkhilfe könnte man sich beispielsweise folgenden Satz einprägen: Ambitioniert versucht Valentin sich für ein Gefühl zu entscheiden. Nachteil ist dabei allerdings, dass bei der Bildung von Assoziationen der einzelnen Teil-Wörter bzw. Silben unter Umständen die Rechtschreibung verloren geht bzw. nicht korrekt rekonstruiert werden kann.

Die Zahlen-Form-Methode eignet sich gut zum einprägen kurzer Zahlenfolgen. Als konkretes Beispiel dient die eigene Matrikelnummer. Allerdings skaliert diese Methode nur sehr schlecht, da der Umfang der sich einzuprägenden Geschichte deutlich mit der Zunahme der abzubildenden Ziffern wächst.

Die Assoziationsmethode ist daher als bevorzugte Methode zu betrachten.
