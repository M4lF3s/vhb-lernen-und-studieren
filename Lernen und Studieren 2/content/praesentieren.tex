% !TEX root = ../lust-1.tex
%
\chapter{Präsentieren}
\label{sec:praesentieren}

\paragraph{Teil 6 der Hausarbeit}
Ein Freund kommt zu Ihnen und sucht Rat. Er muss im Studium eine Präsentation vorbereiten. Da er das noch nie gemacht hat, weiß er nicht genau, wie er die Sache angehen soll und was dabei wichtig ist. Welchen Rat geben Sie ihm? Nehmen Sie Bezug auf die im Modul bearbeiteten Inhalte. \\[0.4em]

Grundsätzlich sollte eine Präsentation unter Anderem die folgenden Punkte berücksichtigen:

\begin{itemize}
    \item Verständlichkeit: \\ Der Inhalt der Präsentation sollte leicht verständlich dargestellt werden. Jedoch sollte die Zuhörerschaft sich auch nicht langweilen. Triviale Inhalte können als solche in der Präsentation markiert werden. Eventuell können Sachverhalte während der Präsentation  übersprungen werden, sollten die Zuhörer der Trivialität auf Nachfrage zustimmen.
    \item Relevanz: \\ Idealerweise wird der Inhalt mit einer gewissen Praxisnähe aufbereitet. Aktuelle Beispiele aus dem Forschungsgebiet oder dem Alltag eignen sich an dieser Stelle besonders.
    \item Länge: \\ Eine Präsentation soll einen Überblick zu einem bestimmten Themengebiet darstellen, daher ist in jedem Fall auf eine angemessene Länge der Präsentation zu achten. Nach 45 min. sollte spätestens eine Pause erfolgen.
\end{itemize}

Für den Inhalt der Präsentation ist es wichtig ein konkretes Ziel zu formulieren. Man sollte sich also die Frage stellen, welche Absicht man mit der Präsentation verfolgt. Abgleichen kann man dies anschließend mit den Zielen der Adressaten. Welches Vorwissen besitzen die Teilnehmer? Was wird von dem Vortrag erwartet? In welcher Form sind die Hörer von der Präsentation betroffen?

Bei der Darstellung ist zu berücksichtigen, dass die zu vermittelnden Informationen auf die wesentlichen Punkte komprimiert und entsprechend visualisiert werden. Diagramme und Bilder sind hier der reinen textuellen Form zu bevorzugen.

Um möglichst frühzeitiges Feedback zu bekommen eignet es sich bereits bei der Erstellung der Folien einzelne Teile zu proben und Kommilitonen zu präsentieren. Man kann dadurch sehr gut testen, ob ein bestimmter Inhaltsteil in der vorgegebenen Zeit verständlich wird oder nicht.
Außerdem sollte die Präsentation einer Generalprobe unterzogen werden um sicher zu stellen, dass auch die technischen Aspekte wie gewollt reibungsfrei funktionieren.
