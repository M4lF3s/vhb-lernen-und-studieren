% !TEX root = ../lust-1.tex
%
\chapter{Literatursuche}
\label{sec:literatursuche}

\paragraph{Teil 3 der Hausarbeit}
Stellen Sie sich vor, dass Sie Literatur zum Thema „Emotionen und Lernen” für eine Hausarbeit suchen müssen. Wie gehen Sie vor? Welchen Einstieg in die Literaturrecherche wählen Sie? Was müssen Sie beachten? Welche grobe Zeitplanung nehmen Sie vor? Nehmen Sie Bezug auf die im Modul behandelten Aspekte. \\[0.4em]

Als erste Anlaufstelle bieten sich die Suchmaschinen “Google Scholar” \cite{WEB:GOOGLE:SCHOLAR} und die bereits im Kurs erwähnte “Web of Science” \cite{WEB:KNOWLEDGE}
an. Hier kann zunächst nach aktuellen Paper und Veröffentlichungen gesucht werden. Ausgehend davon kann als nächstes die jeweils Referenzierte Literatur als Informationsquelle dienen. 
Handelt es sich dabei um Primärliteratur eignet es sich die Kataloge der Uni-Bibliothek nach eben jenen zu durchsuchen. Sollte man die entsprechende Literatur in einer Bibliothek finden, lohnt es sich unter Umständen auch nebenstehende Bücher im Regal etwas genauer zu betrachten, da diese thematisch eventuell ebenfalls zutreffen.

Bei Paper ist vor allem auf die Aktualität und die Originalität zu achten (Stichwort: Plagiate). Bei Primärliteratur sollte man ggf. zunächst die Relevanz prüfen und ob der Beschriebene Inhalt noch dem Stand der Wissenschaft entspricht.

Zum abstecken eines zeitlichen Horizonts kann zunächst der entsprechende Eintrag im Modulhandbuch dienlich sein. Anhand des dort beschriebenen maximalen Zeitaufwandes lässt sich anschließend die im Kurs erwähnte Einteilung nach den “Stufen des Wissenschaftlichen Arbeitens” (Burchert \& Sohr, 2005) vornehmen. \cite{WEB:VHB:LuSt1:Inhaltliche-Aspekte}
