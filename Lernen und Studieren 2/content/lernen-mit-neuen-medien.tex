% !TEX root = ../lust-1.tex
%
\chapter{Lernen mit neuen Medien}
\label{sec:lernen-mit-neuen-medien}

\paragraph{Teil 2 der Hausarbeit}
Entwerfen Sie ein Beispiel für Game-based-learning und beschreiben Sie es. \\[0.4em]

Das Beispiel konzentriert sich auf die “analoge” Variante des Game-based-learning und richtet sich primär an Kinder im Grundschulalter. Ziel des Spiels ist die Lösung einfacher mathematischer Rechenaufgaben, welche durch eine sportliche Wettkampf- bzw. Spielsituation motiviert werden sollen.
Gespielt wird in 2 gleich großen Gruppen. Jede Gruppe bekommt einen Gegenstand (bspw. einen Staffelholz) welches abwechselnd in einem definierten Bereich abgelegt, bzw. aufgenommen und übergeben werden muss. 
Zum Ablauf werden nun den beiden Gruppen Rechenaufgaben gestellt. Sobald diese von dem Kind, welches gerade innerhalb der Gruppe an der Reihe ist, gelöst wurde, darf das Kind los sprinten und entweder das Staffelholz im entsprechenden Bereich ablegen, oder (wenn das Staffelholz bereits abgelegt wurde) dieses aufnehmen und wieder zur Gruppe zurück bringen.
Nach einer vorher festgelegten Anzahl an durchläufen gewinnt diejenige Gruppe, welche als erste alle Aufgaben gerechnet hat.
