% !TEX root = ../lust-1.tex
%
\chapter{Problemlösen}
\label{sec:problemloesen}

\paragraph{Teil 6 der Hausarbeit}
Ein Freund plant einen Urlaub und hat aber nur ein bestimmtes Reisebudget. Er möchte gerne nach Südamerika fliegen und dort für 3 Wochen verschiedene Orte besuchen. Der Flug nach Südamerika ist bereits sehr teuer und jetzt weiß er nicht, wie er das Problem des begrenzten Budgets am besten lösen kann, um das meiste aus dem Urlaub holen zu können. Er sucht nach Rat bei Ihnen. Analysieren Sie den Problemlöseprozess (Problemlöseschritte) anhand dieses Beispiels. Was würden Sie ihm raten? \\[0.4em]

Der Problemlöseprozess besteht aus den 5 Schritten:

\begin{enumerate}
    \item Ausgangszustand formulieren
    \item Zielzustand definieren
    \item Mögliche Lösungswege erörtern
    \item Entscheidung für einen Weg treffen
    \item Durchführen und Evaluieren der Entscheidung
\end{enumerate}

Der Ausgangszustand beinhaltet die Planung eines Urlaubs, für welchen jedoch nur ein begrenztes Budget zur Verfügung steht.

Zielzustand ist möglichst viel Gegenwert für das bezahlte Geld zu bekommen.

Mögliche Lösungswege wären beispielsweise:

\begin{itemize}
    \item Andere Transportmittel nutzen
    \item Die Reisezeit verkürzen
    \item Das Reiseziel zu ändern
    \item Auf der Reise Geld verdienen (Work \& Travel, Remote-Arbeiten, etc.)
    \item Länger sparen / Reise verschieben
    \item Geld leihen
\end{itemize}

Als Ratschlag würde ich geben entweder die Reisezeit zu verkürzen, d.h. an das entsprechende Budget anzupassen, oder das Reiseziel zu ändern. Bei letzterem würde unter Umständen auch beispielsweise die Möglichkeit bestehen andere Transportmittel zu nutzen.
Sollte keiner der beiden Lösungswege eine Option sein ist die sinnvollste Alternative die Reise zu einem späteren Zeitpunkt und mit mehr (finanziellen) Ressourcen anzutreten.
