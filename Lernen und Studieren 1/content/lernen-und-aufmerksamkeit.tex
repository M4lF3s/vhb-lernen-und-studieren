% !TEX root = ../lust-1.tex
%
\chapter{Lernen und Aufmerksamkeit}
\label{sec:lernen-und-aufmerksamkeit}

\paragraph{Teil 3 der Hausarbeit}
Stellen Sie sich vor, dass ein Freund während der Prüfungsphase zu Ihnen kommt und Rat sucht, da er Probleme hat beim Lernen. Was würden Sie ihm raten und warum? Nehmen Sie hierbei Bezug auf die im Modul besprochenen endogenen Einflüsse und Umwelteinflüsse. \\[0.4em]

Zunächst ist einmal anzuraten, die Ursache der Probleme herauszufinden. Probleme beim Lernen können durch viele (auch unbewusste) Faktoren auftreten.

Ein Grund für Ineffektivität beim Lernen kann beispielsweise Schlafdeprivation sein. Nicht ohne Grund existiert im Arbeitszeitgesetz der Paragraph Nr. 5 nach dem Arbeitnehmer zwischen zwei Arbeitstagen mindestens 11h Ununterbrochene Ruhezeit bekommen müssen. Entsprechend sollte man auch zwischen Lerntagen / Lernphasen genügend Ruhezeit einhalten. Die Menge an benötigten Schlaf variiert von Mensch zu Mensch, sollte im Mittel jedoch zwischen 6 und 8 Stunden pro Nacht (für einen Erwachsenen) betragen.
Des weiteren ist zu beachten, dass Schlaf nicht “angespart” werden kann.

Ein weiterer Einflussfaktor können vorübergehende Erkrankungen sein. Dazu zählen auch physische Erkrankungen wie beispielsweise ein gebrochenes Bein. Auch in diesem Fall kann die Lerneffizienz stark beeinflusst sein. Entsprechend ist dem Erkrankten zu empfehlen sich selbst zunächst auszukurieren.

Die (übermäßige) Einnahme psychoaktiver Substanzen kann ebenfalls zu einbußen der Effektivität beim Lernen führen. Rauschmittel wie Alkohol, Koffein, illegale Suchtdrogen, aber auch gewisse Medikamente besitzen keine aktivierungsfördernde Wirkung, sondern gehen meistens mit einer massiven Dämpfung des Leistungsvermögens einher.

Sollten die genannten endogenen Einflüsse bereits berücksichtig werden können auch (unterbewusst) exogene Einflüsse das Lernen beeinflussen.

Der Wechsel des Lernplatzes könnte beispielsweise eine Steigerung der Effektivität mit sich bringen. Konkret könnte man versuchen statt zu Hause am Schreibtisch zu lernen in die Uni-Bibliothek zu gehen. Der Lernort besitzt viele Faktoren, die den Lernfortschritt beeinflussen (Temperatur, Beleuchtung, etc.). Daher. kann es unter Umständen auch schon helfen zu Hause vom Schreibtisch an den Esstisch zu wechseln (oder umgekehrt).

Eine weitere Möglichkeit ist, mit Hilfe von bestimmten Ritualen, sich mental auf das Lernen vorzubereiten. Dies kann dabei helfen die Umstellung von Freizeit zu Arbeitsphasen zu erleichtern.
