% !TEX root = ../lust-1.tex
%
\chapter{Lernprinzipien}
\label{sec:lernprinzipien}

\paragraph{Teil 4 der Hausarbeit}
Überlegen Sie sich ein Lernbeispiel und beschreiben Sie, wie das Lernen in diesem Beispiel ablief. Nehmen Sie hierbei Bezug auf die im Modul besprochene Lernkurve und ihre Charakteristika. Stellen Sie diesem Beispiel eines gegenüber, bei dem Lernen durch Einsicht vorlag. Was waren Unterschiede und was vielleicht Gemeinsamkeiten? Was hat Ihnen mehr Spaß gemacht und warum? … \\[0.4em]

Ein Beispiel für einen Lernvorgang, welcher der Lernkurve nach Ebbinghaus folgt ist das Erlernen einer neuen Programmiersprache (unter der Voraussetzung, dass man bereits programmieren kann). Hier wurden zunächst essenzielle Teile gelernt. Diese umfassen beispielsweise:

\begin{itemize}
    \item Das erstellen von Variablen / Konstanten
    \item Die Definition von Funktionen / Methoden
    \item Das ausführbar machen (engl. compile) des geschriebenen Quellcodes
\end{itemize}

Somit wird sehr schnell ein großer Fortschritt im Lernprozess erzielt, da an dieser Stelle bereits einfache Funktionalität umgesetzt werden kann.

Im Anschluss wurde mehr über die Struktur und den Aufbau der Sprache gelernt. Besonders interessant waren hierbei die möglichen Programmierparadigmen (Objektorientierung, Funktionalität, Aspektorientierung, Prozeduralität, etc.). Dazu kommt das Lernen von Erweiterbarkeit und wie Bibliotheken (also Code anderer Entwickler) genutzt werden können, um komplexere Funktionalitäten umzusetzen, bzw. den eigenen Programmieraufwand zu reduzieren.
Hier fängt der Lernfortschritt an zu stagnieren.

Mit genügend Übung können zum Schluss komplette Anwendungen mit komplexer Funktionalität implementiert werden, wobei “nur” noch Maßnahmen zur Speicher- und Laufzeitoptimierung gelernt werden. Der Fortschritt ist also nur noch sehr gering.

Anders verläuft der Lernvorgang beim verstehen des Induktionsbeweises in der Mathematik.
Hier mussten zunächst etliche Beispielaufgaben betrachtet werden, um die formale Struktur des Beweises nachzuvollziehen bzw. zu verstehen. Die Reproduktion des Beweises ist danach allerdings unabhängig von der Problemstellung möglich.

Die beiden Beispiele unterscheiden sich darin, dass bei ersterem das Lernen schrittweise erfolgt ist, während bei letzterem das Gesamtkonzept verstanden werden musste. Ein weiterer Unterschied ist, dass durch das Lernen einer wieder neuen Programmiersprache die Syntax der alten schneller verdrängt wird, wohingegen der Induktionsbeweis trotz lernen anderer Beweistechniken reproduzierbar bleibt.

Als Gemeinsamkeit lässt sich festhalten, dass beide Lernbeispiele zunächst geübt werden mussten um sie zu verinnerlichen. Außerdem wurde das Lernen beider Beispiele durch Hilfsmittel wie Videos und Vorlesungsskripte unterstützt.

Grundsätzlich hat das lernen der Programmiersprache mehr Spaß gemacht, da hier schrittweise kleine Zwischenerfolge erzielt wurden. Außerdem motiviert die Tatsache immer noch eine Kleinigkeit dazu lernen zu können.
