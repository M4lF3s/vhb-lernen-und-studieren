% !TEX root = ../lust-1.tex
%
\chapter{Lernen und Motivation}
\label{sec:lernen-und-motivation}

\paragraph{Teil 1 der Hausarbeit}
Analysieren Sie Ihr Anspruchsniveau für ein von Ihnen gewähltes Beispiel eines Leistungsbereiches. Dies kann in z.B. einer Vorlesung in der Uni sein, im Sport oder jedem anderen Leistungsbereich, der Ihnen in den Sinn kommt. \\[0.4em]

Zur Analyse eines eigenen Anspruchsniveaus wird im Folgenden der Leistungsbereich der (Software-) Entwicklung betrachtet. Grund dafür ist, dass dieser einen integralen Bestandteil sowohl der alltäglichen Arbeit, als auch des Studiums der Informatik darstellt.
Als konkretes Beispiel dient die Aufgabe der Neugestaltung des Webauftritts unserer Fachschaftsvertretung.

Wie im Online-Kurs vorgeschlagen sollen also in Bezug auf die genannte Themenstellung  die folgenden Fragen beantwortet werden \cite{WEB:VHB:LuSt1:Anspruchsniveau}
\begin{itemize}
    \item Welches Anspruchsniveau habe ich gegenüber welchem Leistungsbereich?
    \item Wie ist es entstanden?
    \item Welche objektiven Vergleiche zu eigenen und fremden Leistungen auf diesem Gebiet habe ich?
    \item Welche leistungsbezogenen Einflüsse auf meine Erwartungen gibt es hier?
    \item Welches subjektive Gewicht messe ich dem Bereich im Vergleich zu anderen Bereichen zu?
\end{itemize}

Die alte Webseite musste aufgrund der veralteten Systeme und den damit einhergehenden Sicherheitslücken zunächst abgeschaltet werden. Dies hatte zur Folge, dass eine Entscheidung zwischen den beiden Folgenden Optionen getroffen werden musste.
Eine Möglichkeit wäre gewesen, updates für die bereits vorhandenen Systeme einzuspielen und den vorhandenen Quellcode dabei so weit zu patchen, damit dieser in der erneuerten Umgebung wieder lauffähig wird. Da es sich bei dem existierendem Quellcode selbst jedoch um keine Eigenentwicklung handelte, sondern um kopierte Code-Schnipsel aus anderen Projekten und Vorlagen, wurde der Aufwand hier mindestens genauso hoch geschätzt wie der einer Neuentwicklung. Auf letztere Variante wurde sich dann, auch im Hinblick auf die Tatsache, dass die notwendigen Fähigkeiten bereits vorhanden sind, entsprechend auch geeinigt.

Die persönliche Motivation dieses Projekt zu übernehmen war zum Einen die eben schon erwähnte Möglichkeit das eigene Know-How mit einbringen zu können. Zum Anderen handelte es sich aber auch um eine sehr gute Möglichkeit neue Technologien kennen zu lernen und somit das eigene Wissen sogar noch zu erweitern. Der Anspruch war, neue Features, die mit dem alten System entweder gar nicht oder nur schwer umzusetzen gewesen wären (bspw. eine Online-Suche für Vorlesungsskripte und Altklausuren, sodass die Sudierenden nicht mehr gezwungen sind wie bisher persönlich im Fachschaftsbüro vorbei zu kommen, oder auch eine Vorschau auf die noch kommenden Uni-Kino Vorführungen), bei der jetzigen Implementierung direkt mit zu berücksichtigen bzw. einzubauen. Die Folge dessen ist ein deutlich geringerer Entwicklungsaufwand in der Zukunft und eine einfachere Bedienbarkeit der Seite.

Als Vergleich können hier die Homepages der Kommilitonen aus anderen Projekten dienen.

Das eigene Anspruchsniveau ist in diesem Fall relativ hoch, da bereits eine langjährige Erfahrung sowohl durch Studium, als auch Arbeit vorhanden ist. Gleichzeitig existiert dadurch aber auch ein deutlich größeres Interesse im Vergleich zu anderen Leistungsbereichen.

Durch die relativ geringe Priorität im Gegensatz zu Studium und Arbeit spielt die detaillierte  Zeitplanung auf Feature-Ebene eine nicht unwichtige Rolle. Dies hat sich besonders gezeigt, als die Weiterentwicklung der Seite aufgrund von Klausuren und Abschlussarbeit aufgeschoben werden musste. Allerdings haben sich auch durch das Lösen einzelner Probleme in der Implementierung immer wieder einzelne Erfolgserlebnisse einstellen lassen können.