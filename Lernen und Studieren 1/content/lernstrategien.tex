% !TEX root = ../lust-1.tex
%
\chapter{Lernstrategien}
\label{sec:lernstrategien}

\paragraph{Teil 5 der Hausarbeit}
Welche Erfahrungen haben Sie bereits bezüglich Erfahrung und Lernen gemacht? Gab es hier positive und negative Beispiele, wie Erfahrung Ihr Lernen beeinflusst hat? Welche Lernstrategien nutzen Sie am häufigsten und wie effektiv sind diese Strategien für Sie? Nehmen Sie hierbei Bezug auf die im Modul bearbeiteten Inhalte. \\[0.4em]

Zu den bevorzugten Lernstrategien zählt zunächst das nutzen des Vorwissens um evt. komplexere Sachverhalte eigenständig herleiten zu können. Mathematische Formeln müssen dadurch also nicht (oder nur zum Teil) auswendig gelernt werden. Zusätzlich ermöglicht das bereits vorhandene Wissen die Bildung von Eselsbrücken, um sich andere / fremde Inhalte einfacher einprägen zu können.

Am häufigsten werden jedoch Notizen zum Lernen genutzt. Diese beinhalten Stichpunktartig die wichtigsten Informationen, komprimiert auf möglichst wenig Zeichen. Ein wichtiger Faktor ist, dass diese Zusammenfassungen handschriftlich und nicht maschinell erstellt werden, da dies den Lernprozess bereits vorantreibt. Unterstützt werden können die Notizen durch entsprechende Grafische Anordnungen und Diagramme.
