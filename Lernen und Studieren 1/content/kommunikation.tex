% !TEX root = ../lust-1.tex
%
\chapter{Kommunikation}
\label{sec:kommunikation}

\paragraph{Teil 7 der Hausarbeit}
Stellen Sie sich vor, Sie sind in sozialen Situationen sehr schüchtern und würden gerne die Kommunikation in sozialen Situationen gestalten. Welche Techniken könnten Sie hier anwenden? Beschreiben Sie diese konkret an Beispielen. Nehmen Sie Bezug zum im Modul behandelten Stoff. \\[0.4em]

Die Zielorientierung ist ungemein wichtig in der Kommunikationsgestaltung. Erreicht werden kann dies durch konkretes ausformulieren des Ziels und anschließendem Mitteilen an den Kommunikationspartner. Ein konkretes Beispiel könnte sein, dass man mit einem Kommilitonen ein Gespräch über Probleme mit dem eigenen Partner führen möchte.
In diesem Fall ist es beispielsweise zielführender das Gespräch zu beginnen mit “Ich habe ein Problem und hätte gerne einen Ratschlag von dir” als sich einfach aus heiterem Himmel über den Partner zu beschweren. Der Kommunikationspartner weiß bei ersterem schon vor dem eigentlichen Inhalt der Kommunikation was von ihm erwartet wird.

Als weiterer wichtiger Punkt gilt die Reziprozität, also das Gleichgewicht der Redeanteile. Dies kann unter Anderem durch Nachfragen erzielt werden.
Vorstellen könnte man sich beispielsweise ein Gruppengespräch über ein Thema, welches für einen selbst noch relativ neu ist. Bevor man in diesem Fall mit Halbwissen in das Gespräch versucht einzubringen ist es sinnvoller bei Unklarheiten nachzufragen.

Ein dritter Faktor sind die sogenannten “Ich”-Formulierungen. Diese sollen die eigene Meinung in einem Gespräch widerspiegeln. Ein Beispiel hierfür wäre eine Gruppenarbeit bei der man mit der Leistung des Kommilitonen unzufrieden ist und diesen entsprechend darauf hinweisen möchte. Anstatt die Konversation zu beginnen mit “Das hättest du detaillierter formulieren können” könnte man sich besser wie folgt ausdrücken “Ich hätte diesen Teil etwas detaillierter formuliert”.
